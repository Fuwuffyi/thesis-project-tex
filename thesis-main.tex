\documentclass[12pt,a4paper,openright,twoside]{book}
\usepackage[utf8]{inputenc}
\usepackage{disi-thesis}
\usepackage{code-lstlistings}
\usepackage{notes}
\usepackage{shortcuts}
\usepackage{acronym}

\school{\unibo}
\programme{Corso di Laurea in Ingegneria e Scienze Informatiche}
\title{OpenGL e Vulkan: un caso di studio}
\author{Palazzini Luca}
\date{\today}
\subject{Computer Graphics}
\supervisor{Prof.ssa Damiana Lazzaro}
\session{I}
\academicyear{2024-2025}

% TODO: Definition of acronyms
\acrodef{API}{Application Programming Interface}
\acrodef{PBR}{Physically Based Rendering}

\mainlinespacing{1.241}

\begin{document}

\frontmatter\frontispiece

% TODO: Write abstract
\begin{abstract}	
Max 2000 characters, strict.
\end{abstract}

% TODO: Write dedication (optional)
\begin{dedication}
Optional. Max a few lines.
\end{dedication}

% Table of contents
\tableofcontents   
% TODO: Uncomment when image added \listoffigures
% TODO: Uncomment when listing added \lstlistoflistings

% Main content
\mainmatter

%----------------------------------------------------------------------------------------
\chapterWithoutNumber{Introduzione}
\label{chap:introduction}
%----------------------------------------------------------------------------------------

\section{Contesto e Motivazioni}
Negli ultimi anni, il campo della grafica computazionale ha visto un'evoluzione significativa delle \emph{Graphics \ac{API}} verso modelli di programmazione più vicini all'hardware e orientati alle alte prestazioni.  
OpenGL, storicamente una delle \ac{API} più diffuse, offre un modello di programmazione di tipo \emph{high-level}, che semplifica lo sviluppo ma lascia al driver una gestione implicita di numerosi aspetti, come la sincronizzazione e la gestione della memoria.  
Vulkan, introdotta dal \emph{Khronos Group} nel 2016, adotta invece un approccio \emph{low-level}, demandando al programmatore un controllo esplicito sulle risorse e sull’esecuzione dei comandi, con l’obiettivo di ridurre l’overhead del driver e di consentire una migliore parallelizzazione del carico di lavoro.

Lo scopo di questo elaborato è indagare le differenze prestazionali e architetturali tra le due \ac{API}, attraverso la realizzazione di un motore di rendering capace di utilizzare entrambe le tecnologie.  
Il confronto si concentra in particolare sull’impatto del multithreading in Vulkan rispetto al modello single-thread tipico di OpenGL, analizzando come la differente gestione della pipeline grafica influisca sulle prestazioni complessive.

\section{Objectives}

\section{Methodology}

\paragraph{Structure of the Thesis}

%----------------------------------------------------------------------------------------
\chapter{Background}
\label{chap:background}
%----------------------------------------------------------------------------------------

\section{Rendering Pipelines}

\section{Physically Based Rendering (PBR)}

\section{Graphics APIs: OpenGL and Vulkan}

\section{Libraries and Tools}

\section{Related Work}

%----------------------------------------------------------------------------------------
\chapter{Requirements and Analysis}
\label{chap:analysis}
%----------------------------------------------------------------------------------------

\section{Functional Requirements}

\section{Non-functional Requirements}

\section{Constraints}

\section{Evaluation Strategy}

%----------------------------------------------------------------------------------------
\chapter{Design and Architecture}
\label{chap:design}
%----------------------------------------------------------------------------------------

\section{System Overview}

\section{Object-Oriented Design}

\section{Renderer Architecture}

\section{Deferred Rendering Pipeline}

\section{Resource Management}

%----------------------------------------------------------------------------------------
\chapter{Implementation}
\label{chap:implementation}
%----------------------------------------------------------------------------------------

\section{Codebase Structure}

\section{Key Components}

\section{PBR Material System}

\section{Particle Simulation}

\section{Performance GUI}

%----------------------------------------------------------------------------------------
\chapter{Performance Evaluation}
\label{chap:evaluation}
%----------------------------------------------------------------------------------------

\section{Experimental Setup}

\section{Results}

\section{Discussion}

%----------------------------------------------------------------------------------------
\chapterWithoutNumber{Conclusions and Future Work}
\label{chap:conclusions}
%----------------------------------------------------------------------------------------

\section{Summary}

\section{Findings}

\section{Limitations}

\section{Future Work}

% End of thesis
\backmatter

\nocite{*} % TODO: Comment when cited book

\bibliographystyle{alpha}
\bibliography{bibliography}

% TODO: Write acknowledgements (optional)
\begin{acknowledgements}
Optional. Max 1 page.
\end{acknowledgements}

\end{document}
