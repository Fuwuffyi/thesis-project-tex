\documentclass[12pt,a4paper,openright,twoside]{book}
\usepackage[utf8]{inputenc}
\usepackage{disi-thesis}
\usepackage{code-lstlistings}
\usepackage{notes}
\usepackage{shortcuts}
\usepackage{acronym}

\school{\unibo}
\programme{Corso di Laurea in Ingegneria e Scienze Informatiche}
\title{OpenGL e Vulkan: un caso di studio}
\author{Palazzini Luca}
\date{\today}
\subject{Computer Graphics}
\supervisor{Prof.ssa Damiana Lazzaro}
\session{I}
\academicyear{2024-2025}

% Definition of acronyms
\begin{acronym}[API]
  \acro{API}{Application Programming Interface}
  \acro{PBR}{Physically Based Rendering}
  \acro{CPU}{Central Processing Unit}
  \acro{GPU}{Graphical Processing Unit}
\end{acronym}

\mainlinespacing{1.241}

\begin{document}

\frontmatter\frontispiece

% TODO: Write abstract
\begin{abstract}	
Max 2000 characters, strict.
\end{abstract}

% TODO: Write dedication (optional)
\begin{dedication}
Optional. Max a few lines.
\end{dedication}

% Table of contents
\tableofcontents   
% TODO: Uncomment when image added \listoffigures
% TODO: Uncomment when listing added \lstlistoflistings

% Main content
\mainmatter

%----------------------------------------------------------------------------------------
\chapterWithoutNumber{Introduzione}
\label{chap:introduction}
%----------------------------------------------------------------------------------------

\paragraph{Contesto e Motivazioni}
Negli ultimi anni, il campo della grafica computazionale ha visto un'evoluzione significativa delle \emph{Graphics \acs{API}}
verso modelli di programmazione più vicini all'hardware e orientati alle alte prestazioni.  
OpenGL, storicamente una delle \ac{API} più diffuse, offre un modello di programmazione di tipo \emph{high-level}, che
semplifica lo sviluppo ma lascia al driver una gestione implicita di numerosi aspetti, come la sincronizzazione e la
gestione della memoria.  
Vulkan, introdotta dal \emph{Khronos Group} nel 2016, adotta invece un approccio \emph{low-level}, demandand
programmatore un controllo esplicito sulle risorse e sull’esecuzione dei comandi, con l’obiettivo di ridurre l’overhead
del driver e di consentire una migliore parallelizzazione del carico di lavoro.

Lo scopo di questo elaborato è indagare le differenze prestazionali e architetturali tra le due \ac{API}, attraverso
la realizzazione di un motore di rendering capace di utilizzare entrambe le tecnologie.  
Il confronto si concentra in particolare sull’impatto del multithreading in Vulkan rispetto al modello single-thread
tipico di OpenGL, analizzando come la differente gestione della pipeline grafica influisca sulle prestazioni complessive.

\paragraph{Obiettivi}
L’obiettivo principale del lavoro è progettare e sviluppare un motore di rendering scritto in \emph{C++23},
in grado di operare sia con OpenGL 4.6 sia con Vulkan 1.4.
Il motore implementa un approccio di tipo \emph{deferred rendering}, supporta materiali \ac{PBR}, luci direzionali,
puntuali e spot, e include ulteriori passaggi di rendering per particelle e oggetti di debug.
L’architettura segue un paradigma orientato agli oggetti, integrando librerie e framework comuni.
L’obiettivo sperimentale è valutare, a parità di contenuto e condizioni di rendering, l’efficienza delle due API 
in termini di:
\begin{itemize}
    \item tempo medio per frame e frame rate;
    \item utilizzo della \acs{CPU} e della \acs{GPU};
\end{itemize}

\paragraph{Metodo di lavoro}
Il progetto è stato sviluppato con un approccio incrementale, secondo cicli iterativi di implementazione e validazione.  
Dopo la fase di progettazione architetturale, il motore è stato realizzato con un sistema modulare che separa la logica
di rendering dal resto della gestione della scena, consentendo di confrontare in modo diretto i due backend grafici.  
% TODO: mention Sponza scene %
I test prestazionali sono stati condotti su una scena principale, utilizzando diversi hardware, variando numero di luci
e \emph{particle systems}, per misurare in modo oggettivo i benefici derivati dal parallelismo offerto da Vulkan.

\paragraph{Struttura del documento}
Il documento è organizzato come segue:
\begin{itemize}
   \item Il \textbf{Capitolo 1} introduce i concetti fondamentali relativi al rendering, alle \ac{API} grafiche moderne e alle tecniche \ac{PBR} e deferred rendering;
   \item Il \textbf{Capitolo 2} analizza i requisiti del progetto e le scelte progettuali alla base del motore sviluppato;
   \item Il \textbf{Capitolo 3} descrive l’architettura del sistema e i principali componenti software;
   \item Il \textbf{Capitolo 4} illustra l’implementazione e mostra esempi di codice e schermate del motore in funzione;
   \item Il \textbf{Capitolo 5} presenta la valutazione sperimentale, i risultati delle misure e la loro analisi critica.
\end{itemize}

%----------------------------------------------------------------------------------------
\chapter{Background}
\label{chap:background}
%----------------------------------------------------------------------------------------

\section{Rendering Pipelines}

\section{Physically Based Rendering (PBR)}

\section{Graphics APIs: OpenGL and Vulkan}

\section{Libraries and Tools}

\section{Related Work}

%----------------------------------------------------------------------------------------
\chapter{Requirements and Analysis}
\label{chap:analysis}
%----------------------------------------------------------------------------------------

\section{Functional Requirements}

\section{Non-functional Requirements}

\section{Constraints}

\section{Evaluation Strategy}

%----------------------------------------------------------------------------------------
\chapter{Design and Architecture}
\label{chap:design}
%----------------------------------------------------------------------------------------

\section{System Overview}

\section{Object-Oriented Design}

\section{Renderer Architecture}

\section{Deferred Rendering Pipeline}

\section{Resource Management}

%----------------------------------------------------------------------------------------
\chapter{Implementation}
\label{chap:implementation}
%----------------------------------------------------------------------------------------

\section{Codebase Structure}

\section{Key Components}

\section{PBR Material System}

\section{Particle Simulation}

\section{Performance GUI}

%----------------------------------------------------------------------------------------
\chapter{Performance Evaluation}
\label{chap:evaluation}
%----------------------------------------------------------------------------------------

\section{Experimental Setup}

\section{Results}

\section{Discussion}

%----------------------------------------------------------------------------------------
\chapterWithoutNumber{Conclusions and Future Work}
\label{chap:conclusions}
%----------------------------------------------------------------------------------------

\paragraph{Summary}

\paragraph{Findings}

\paragraph{Limitations}

\paragraph{Future Work}

% End of thesis
\backmatter

\nocite{*} % TODO: Comment when cited book

\bibliographystyle{alpha}
\bibliography{bibliography}

% TODO: Write acknowledgements (optional)
\begin{acknowledgements}
Optional. Max 1 page.
\end{acknowledgements}

\end{document}
